\chapter{DESARROLLO DEL TRABAJO}

\section{Backup y Restauración}
\subsection{RMAN}
El Recovery Manager (RMAN) es una utilidad usada para respaldar (backup), restaurar, recuperar y clonar bases de datos ORACLE.

Este producto se encarga de la gestión de backups y restauración de data files, archive logs y control files, además de poder ser usado para la recuperación completa o incompleta de una Base de datos.

Rman tiene la característica de ser configurado de dos formas , la primera, más limitada y con menos opciones , que solo puede gestionar una sola base de datos y donde toda la información de los backups es guardada en el controlfile y la segunda, más completa y robusta, manejado por un repositorio que se guarda en la base de datos en forma de esquema y que nos permitirá la gestión de backups de un mayor número de instancias.

RMAN puede ser operado desde Oracle Enterprise Manager o desde linea de comandos.La mayor ventaja de RMAN es que sólo se utiliza el espacio de copia de seguridad en la base de datos.

RMAN nos permite realizar backup ya sea en frío o en caliente.

Ejemplo de RMAN:

\begin{list}{}{}
\item [oracle@localhost oracle] \$ rman
\item Recovery Manager: Release 10.1.0.2.0 - Production Copyright (c) 1995, 2004, Oracle. All rights reserved.
\item RMAN> connect target;
\item connected to target database: ORCL (DBID=1058957020)
\item RMAN> backup database;
\end{list}


\subsection{Backups de la BD en Frio}
Los backups en frio implican parar la Base de Datos en modo normal y copiar todos los ficheros sobre los que se asienta(datafile,controlfile y logfile). Antes de parar la Base de Datos hay que parar también todos las aplicaciones que estén trabajando con la Base de Datos. Una vez realizada la copia de los ficheros, la Base de Datos se puede volver a arrancar.

Los pasos que hay que seguir para realizar un backup en frió(en oracle) serían:

\begin{enumerate}
\item Conocer y listar la ubicación de los datafiles, controlfiles y logfiles. Esto se hace ejecutando:
\item Detener la base de datos mediante shutdown normal o inmediato.
\item Copiar los archivos datafiles, controlfiles y logfiles a un medio de backup preferido como cinta, disco duro, otra máquina, etc.
\end{enumerate}

\section{Antecedentes}
Para conocer cómo ha ido evolucionando la Información, las Tecnologías de Información y Comunicación, y las Teorías sobre Gestión de la Información, se presentan a continuación los eventos m\'as importantes ocurridos a largo de nuestra historia a partir del año 500 después de Cristo

\begin{enumerate}
\item Priemr Antecedente
\item Segundo Antecedente
\end{enumerate}

\section{Estado del Arte}
Se entiende por Información al conjunto de datos que han sido procesados con la finalidad de establecer un mensaje y generar conocimiento del sistema que lo reciba. El dato es su unidad mínima, el cual por sí solo no posee ningún valor, pero en conjunto genera información. Esta información al ser organizada adecuadamente se convierte en conocimiento y luego del resultado de su análisis se convierte en finalmente sabiduría. \citeA{Blanco2013}

\subsection{Estrategias de Respaldo}

\begin{equation}
f(x)= \left\{ \begin{array}{l||c|l}
x^2 & \mbox{ si } & x<0 \\ \hline
& & \\
x-1 & \mbox{ si } & x>0
\end{array}
\right.	
\end{equation}

\begin{table}[ht]
\centering{
    \fontfamily{ptm}
        \selectfont{
            \rowcolors{1}{gray}{cyan}
            \begin{tabular}{ll}
                $x_{n+1}$ & $|x_{n+1}-x_n|$\\ \hline
                1.20499955540054 & 0.295000445\\
                1.17678931926590 & 0.028210236\\
                1.17650196994274 & 0.000287349\\
                1.17650193990183 & 3.004$\times10^{-8}$\\
                1.17650193990183 & 4.440$\times10^{-16}$\\ \hline
            \end{tabular}
}}
\caption{Iteracion de Newton para $x^2-\cos(x)-1=0$ con $x_0=1.5$}
\end{table}


Este un ejemplo de imagen

\begin{figure}[ht]
\centering
\includegraphics[scale=1.7]{images/universe.jpg}
\caption{The Universe}
\label{fig:universe}
\end{figure}

Terminamos

\subsection{Estrategias de Recuperaci\'on}

\begin{table}[ht]
\centering{
    \fontfamily{ptm}
        \selectfont{
            %\rowcolors{1}{gray}{cyan}
            \begin{tabular}{ll}
                Actividad & Duraci\'on\\ \hline
                Elaboración de los Aspectos Generales del Trabajo de Investigaci\'on   &   10 d\'ias\\
                Elaboración del Marco Te\'orico   &   35 d\'ias\\
                Elaboración del Marco Metodol\'ogico   &   15 d\'ias\\
                Elaboración del Marco Metodol\'ogico   &   15 d\'ias\\
                1.17650193990183 & 4.440$\times10^{-16}$\\ \hline
            \end{tabular}
}}
\caption{Cronograma}
\end{table}