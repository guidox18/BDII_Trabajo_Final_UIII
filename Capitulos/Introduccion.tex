\chapter*{INTRODUCCIÓN}
    \addcontentsline{toc}{chapter}{INTRODUCCI\'ON}

Como parte del contenido del curso de Sistemas de Bases de Datos 2, se tocan temas como lo son la realización de Backups y sus respectivas Restauraciones en una base de datos, adentrándonos mas en el tema se tocan específicamente lo que es la realización de dichas operaciones asociadas a las copias de seguridad tanto en caliente como en frio, así como haciendo uso de herramientas como el RMAN y sin hacer uso de ellas mediante una consola de SQLPLUS.

Por lo que el tema a desarrollar en este documento trata de cómo realizar las operaciones descritas anteriormente dejando constancia de cada una de ellas mediante esta documentación la cual será la base para futuras implementaciones de copias de seguridad asociadas a Oracle.
Así como también se presenta un pequeño Marco teórico objeto de investigación bajo diferentes fuentes bibliográficas que nos han servido para enriquecer el conocimiento y teniendo como finalidad la elaboración exitosa de este documento.